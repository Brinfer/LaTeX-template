%-----------------------------------------------------------------
% LaTeX template for a document
% Written By: Pierre-Louis GAUTIER
% Date Updated: August 21, 2021 (v1.2.1)
%-----------------------------------------------------------------

\documentclass[a4paper, 11pt]{article}

% NOTE To conserve the space in the sentence passed as option to the package double braces to protect the spaces.
% It's an unfortunate feature of the core latex option handling

\usepackage[imageHeaderLeft     =   {figures/logo/logoCompany.png},
            subject             =   {{Subject of the document, here it is an exemple, displayed in the metadata of the document}},
            keywords            =   {{Example, Pierre-Louis, GAUTIER}},
            encoding            =   {utf8},
            language            =   {english},
            showunnemberred     =   true
            ]{Parameter}

\usepackage[logoCenter      =   {figures/logo/logoCompany.png},
            logoBottom      =   {figures/logo/logoLatex.png},
            nameCompany     =   {{Company Name}}
            ]{FrontPage}

\usepackage{lipsum} % TODO generate fake text, remove it

%%%%%%%%%%%%%%%%%%%%%%%%%%%%%%%%%%%%%%%%%%%%%%%%%%%%%%
%% General config for header, footer and front page %%

\author{Pierre-Louis GAUTIER}
\title{Title}
\date{\normalsize\today} % Enter a custom date or let it, it's the current date

%% Load the glossary from the given files %%

\newignoredglossary{ignored}

\makeglossaries
\loadglsentries{glossary.tex}
\loadglsentries[ignored]{glossary-ignored.tex}
\loadglsentries{abbreviation.tex}

%% Load the bibliography from the given files %%

\addbibresource{biblio.bib}

%%%%%%%%%%%%%%%%%%%%%%%%%%%%%%%%%%%%%%%%%%%%%%%%%%%%%%

\begin{document}

\maketitle

\newpage
\pagenumbering{arabic} % restart the page numbering
\tableofcontents

\newpage
\part{Presentation}
This document is a simple and quite general template for \gls{latex} document.

\begin{figure}[H]
    \centering
    \includegraphics[width=5cm]{figures/logo/logoLatex.png}
    \caption{\glsentrytext{latex} project logo}
    \label{fig:latexProject}
\end{figure}

\section{Package used}

\begin{itemize}
    \item \href{https://www.ctan.org/pkg/ae}{ae} A set of virtual fonts which emulates T1 coded fonts using the standard CM fonts.
    \item \href{https://www.ctan.org/pkg/babel}{babel} This package manages culturally-determined typographical (and other) rules
          or a wide range of languages.
    \item \href{https://www.ctan.org/pkg/biblatex}{biblatex} A complete reimplementation of the bibliographic facilities provided by \gls{latex}.
    \item \href{https://www.ctan.org/pkg/biblatex}{biblatex} A complete reimplementation of the bibliographic facilities provided by \gls{latex}.
    \item \href{https://www.ctan.org/pkg/csquotes}{csquotes} This package provides advanced facilities for inline and display quotations.
    \item \href{https://www.ctan.org/pkg/fancyhdr}{fancyhdr} The package provides extensive facilities, both for constructing headers
          and footers, and for controlling their use.
    \item \href{https://www.ctan.org/pkg/geometry}{geometry} The package provides an easy and flexible user interface to customize
          page layout, implementing auto-centering and auto-balancing mechanisms.
    \item \href{https://www.ctan.org/pkg/glossaries}{\glsentrytext{gls}} This package provides improvements and extra features
          to the glossaries package.
    \item \href{https://www.ctan.org/pkg/graphicx}{graphicx} The package builds upon the graphics package, providing a key-value
          interface for optional arguments to the \verb=\includegraphics= command.
    \item \href{https://www.ctan.org/pkg/hyperref}{hyperref} The \textbf{hyperref} package is used to handle cross-referencing commands in
          \gls{latex} to produce hypertext links in the document.
    \item \href{https://www.ctan.org/pkg/kvoptions}{kvoption} This package offers support for package authors who want to use options
          in key-value format for their package options.
    \item \href{https://www.ctan.org/pkg/lastpage}{lastpage} Reference the number of pages in your \gls{latex} document through the
          introduction of a new label.
    \item \href{https://www.ctan.org/tex-archive/info/lmodern}{lmodern} Provide some symbol.
    \item \href{https://www.ctan.org/pkg/minted}{minted} The package that facilitates expressive syntax highlighting in \gls{latex}.
    \item \href{https://www.ctan.org/pkg/multirow}{multirow} The package has a lot of flexibility, including an option for specifying
          an entry at the “natural” width of its text.
    \item \href{https://www.ctan.org/pkg/tabularx}{tabularx} The package defines an environment \textbf{tabularx}, an extension of
          \textbf{tabular} which has an additional column designator, X, which creates a paragraph-like column whose width automatically
          expands so that the declared width of the environment is filled.
    \item \href{https://www.ctan.org/pkg/titlesec}{titlesec} A package providing an interface to sectioning commands for selection
          from various title styles.
    \item \href{https://www.ctan.org/pkg/xcolor}{xcolor} The package starts from the basic facilities of the \textbf{color} package,
          and provides easy driver-independent access.
\end{itemize}

\subsection{babel}

You can change the language of \textbf{babel} using the \textit{language} option of the \textbf{\gls{parameter}} package. By default, it
is \textit{french}.

\subsection{biblatex}

TODO

\subsection{csquote}

TODO

\subsection{fancyhdr}

You can choose the image to add in the header by using the \textit{imageHeader} option of the \textbf{\gls{parameter}} package.
By default, there is no image.

\subsection{\glsentryname{gls}}

A very practical and customizable package. Refer to the internet to learn more.\newline

Some tutorial:
\begin{itemize}
    \item \href{https://www.dickimaw-books.com/latex/thesis/html/makeglossaries.html}{Dickimaw Books}
    \item \href{https://www.resurchify.com/latex_tutorial/latex_glossaries.php}{Resurchify}
    \item \href{https://fr.overleaf.com/learn/latex/Glossaries}{Overleaf}
\end{itemize}

In case a glossary entry is used for the first time, then a footnote is added giving the description of the entry,
moreover, the hyperlink of this entry is removed.
Each new call of the same glossary entry in the same page will have its hyperlink removed.\newline

In summary, calling a glossary entry for the first time adds a footnote, and prevents any hyperlink from being
added to that entry. If calls to that entry are made to other pages, then only the first call to that entry has a hyperlink.\newline

It is also possible to have entries that are not displayed in the glossary, that will never have a hyperlink.
You just have to specify that the type of the entry is \textit{ignored} or put it in the \underline{glossary-ignored.tex} file.
This input can therefore be used as a variable, but with the full support of \gls{gls}.

\subsection{hyperref}

The metadata of the PDF are modified with this package, the name of the author, the title of the document given
by the \verb=\title= and \verb=\author= commands are added by default. You can add keywords and specify the subject using
\textit{keywords} and \textit{subject} option of the \textbf{\gls{parameter}} package. By default, it is empty.

\subsection{inputenc}

You can specify the encoding of your source files to \textbf{inputc} using the \textit{encoding} option of the \textbf{\gls{parameter}}
package. By default, it is \textit{utf8}.

\subsection{titlesec}

Customize the name of the sections, and add the \verb=\subparagraph= command to have more possibilities. The table of contents
will also display it.\newline

The color of the sections can be recovered with the commands \verb=\color= or \verb=\textcolor= :

\begin{table}[H]
    \centering
    \begin{tabular}{|ccccc|}
        \hline
        \textcolor{sectionColor}{sectionColor}
         & \textcolor{subSectionColor}{subSectionColor}
         & \textcolor{subSubSectionColor}{subSubSectionColor}
         & \textcolor{paragraphColor}{paragraphColor}
         & \textcolor{subparagraphColor}{subparagraphColor}
        \tabularnewline
        \hline
    \end{tabular}
    \caption{The color of the sections}
    \label{ta:sectionColors}
\end{table}

\subsection{Minted}

\subsubsection{Python}
\begin{minted}{Python}
    for i in range(10):
        print("Hello World")
\end{minted}

\subsubsection{C}
\begin{minted}{C}
    for(int i = 0; i < 10; i++) {
        println("Hello World\n");
    }
\end{minted}

\subsubsection{Java}
\begin{minted}{Java}
    class HelloWorld {
        public static void main(String[] args) {
            System.out.println("Hello, World!");
        }
    }
\end{minted}

\section{Available option}

\section{To-do}

\begin{itemize}
    \item Hide the Content section in the Table of content when \textit{showunnemberred} option is set at \verb=true=.
    \item Start a \textbf{part} on a new page when use the command \verb=\part=.
    \item Set up the bibliography.
    \item Be able to past the table of figure and the table of table (and the glossary in same time) as \textbf{subsection} if
          the user want it.
    \item Use \textbf{minted} or \textbf{listing} to highlight code block and inline code.
    \item Use acronym with \textbf{glossaries} package.
    \item Set the label of the list in function of the level.
\end{itemize}


\clearpage
\part*{Part}
\lipsum[1-3]
\section*{Section}
\lipsum[4-6]
Just a sentence to call an entry in the glossaries: \gls{random1}.
\subsection*{Subsection}
\lipsum[7-8]
Just a sentence to call two entries in the glossaries: \gls{random2} and \gls{random3}.
\subsubsection*{Subsubsection}
\lipsum[9]
\paragraph*{Paragraph}
\lipsum[10]
Just a sentence to call four entries in the glossaries: \gls{random1}, \gls{random2}, \gls{random3} and \gls{random4}.\newline
I recall \gls{random1}, \gls{random2} and \gls{random4}.\newline
If you check in the \textit{glossary.tex} only the called entry will appear in the \textbf{Glossary} section.
\subparagraph*{Subparagraph}
\lipsum[11]

\clearpage
\part*{Annex}
\listoffigures
\listoftables

\clearpage
\printglossary % Display the glossary, see the documentation for the options

\clearpage
\printbibliography

\end{document}
